\documentclass{beamer}
\usetheme{Goettingen}


\usepackage{listings}
\setbeamercovered{invisible}
% To remove the navigation symbols from 
% the bottom of slides
\setbeamertemplate{navigation symbols}{} 


\usepackage{graphicx}
\usepackage{verbatim}

\title[A brief introduction to Git]{A brief introduction to Git}
\author{Michael Hull}
\institute[University of Edinburgh]
{
University of Edinburgh \\
\medskip
{\emph{mikehulluk@googlemail.com}}
}
\date{\today}

\begin{document}

\begin{frame}
\titlepage
\end{frame}


\begin{frame}
\frametitle{Overview}
\begin{block}
{This talk}
\begin{itemize}
\item Overview of Git (20 mins)
\item Practical Session (30 mins)
\item Further Concepts (10 mins)
\end{itemize}
\end{block}
\end{frame}



\begin{frame}
\frametitle{How do you ...}

\begin{block}
{ ... keep your Laptop \& Desktop in-sync}
\begin{itemize}
\item Dropbox?
\item E-mail? 
\item Google Drive?
\item USB stick?
\item (Unison/rsync/ssh)
\item ??
\end{itemize}
\end{block}

\pause

\begin{block}
{... backup your code}
\begin{itemize}
\item \emph{(Methods above)} 
\item Time Machine 
\item Copy to university-server
\item ??
\end{itemize}
\end{block}

\end{frame}



\begin{frame}
\frametitle{How do you ...}
\begin{block}
{... keep old copies of your code}	
\begin{itemize}
\item Create 'zip' files of the directories with the date. 
\item ??
\end{itemize}
\end{block}

\pause

\begin{block}
{ \emph{Version Control systems solve these problems (\& more)}	}
\end{block}




\end{frame}


\begin{frame}
\frametitle{Version Control}

\begin{block}
{History}
\begin{itemize}
\item Allow multiple developers to work on the same code (1970's)
\item Designed by programmers for programmers - work well with text-files
\item Commandline \& graphical tools
\end{itemize}
\end{block}

\end{frame}


\begin{frame}
\frametitle{Version Control}


\begin{block}
{Common Concepts}
\begin{itemize}
\item Save \emph{snapshots} of a directory at a point in time, which is associated with a version number. Each snapshot is called a {\bf{commit}}. 
\item Allow you to see the changes made between particular snapshots.
\item Allow you to jump in time to different snapshots - {\bf{check-out}}. 
\item Allow you to {\bf{branch}} and {\bf{merge}}.
\end{itemize}
\end{block}
\end{frame}


\begin{frame}
\frametitle{Centralised vs Distributed}
\begin{block}{}
\begin{description}
\item[Centralised - \emph{SVN (subversion)} ] The project is stored on a central server, users communicate with the server to make snapshots, jump backwards to a previous commit, etc.
\item[Distributed - \emph{git (mercurial,...)} ] Each developer has an entire copy of the project on thier local machine. Changes are 'pushed' and 'pulled' between copies of the project. (Often a central repository is used) (Backups)
\end{description}
\end{block}
\end{frame}
%
\begin{frame}
\frametitle{Git}
\begin{block}{}
\begin{itemize}
\item Modern Distributed Version Control system (2005)
\item Written by Linus Torvald to manage the development of the Linux-Kernel. (Last week 1 commit every 2.5 minutes.)
\item Steep learning curve (there are GUI tools)
\item (git can 'talk-to' SVN repositories)
\item (Windows/Mac/Linux/etc)
\item Although it is designed for collaborative work - we will just look at
using it for a single user locally with no server.
\end{itemize}
\end{block}
\end{frame}


\begin{frame}[fragile]
\frametitle{Git Commands}
%
\begin{block}{Initial Setup}
\begin{itemize}
\item \verb|git init| - used once to start a new project
\item \verb|git clone <location>| - clone an existing project
\end{itemize}
\end{block}


\begin{block}{Status}
\begin{itemize}
\item \verb|git status| - show the status of the files in the repository
\end{itemize}
\end{block}

\end{frame}


\begin{frame}[fragile]
\frametitle{Git Commands}

\begin{block}{Making Commits}
\begin{itemize}
\item \verb|git add <filename>| - select files to add to the next commit (snapshot)
\item \verb|git commit -m '<message>'| - make a commit
\end{itemize}
\end{block}


\begin{block}{Syncronising Commits}
\begin{itemize}
\item \verb|git pull <location>| - update the local repository with commits made in a remote repository.
\item \verb|git push <location>| - update a remote repository with commits made in this local repository.
\end{itemize}
\end{block}
\end{frame}


\begin{frame}[fragile]
\frametitle{Git Tools}

\begin{block}{Graphical Tools}
\begin{itemize}
\item gitg
\item git cola
\end{itemize}
\end{block}
\end{frame}







\begin{frame}[fragile]
\begin{block}{Initial Config - tell git who you are}
\begin{lstlisting}[language=bash,basicstyle=\ttfamily\scriptsize]
$ git config --global user.name "John Doe"
$ git config --global user.email johndoe@example.com
\end{lstlisting}
\end{block}
\end{frame}


\begin{frame}[fragile]
\begin{block}{Starting a new git repository}
\begin{lstlisting}[language=bash,basicstyle=\ttfamily\scriptsize]
% mkdir myproject
% cd myproject
% git init
\end{lstlisting}

\begin{lstlisting}[language=bash,basicstyle=\sl\ttfamily\scriptsize,breaklines=true]
Initialized empty Git repository in /home/mhtest/myproject/.git/.
\end{lstlisting}

\begin{lstlisting}[language=bash,basicstyle=\ttfamily\scriptsize,breaklines=true]
% ls -a
\end{lstlisting}

\begin{lstlisting}[language=bash,basicstyle=\sl\ttfamily\scriptsize]
.  ..  .git
\end{lstlisting}

\end{block}
\end{frame}



\begin{frame}[fragile]
\begin{block}{Look at initial status}
\begin{lstlisting}[language=bash,basicstyle=\ttfamily\scriptsize]
% git status
\end{lstlisting}
\begin{lstlisting}[language=bash,basicstyle=\sl\ttfamily\scriptsize,breaklines=true]
# On branch master
#
# Initial commit
#
nothing to commit (create/copy files and use "git add" to track)

\end{lstlisting}
\end{block}
\end{frame}




\begin{frame}[fragile]
\begin{block}{Create a file}
\begin{lstlisting}[language=bash,basicstyle=\ttfamily\scriptsize]
% gedit myfile.txt
% cat myfile.txt
\end{lstlisting}

\begin{lstlisting}[language=bash,basicstyle=\sl\ttfamily\scriptsize]
Mike's new file
Its not very interesting yet.
\end{lstlisting}
\end{block}

\begin{block}{Check the status again}
\begin{lstlisting}[language=bash,basicstyle=\ttfamily\scriptsize]
% git status
\end{lstlisting}

\begin{lstlisting}[language=bash,basicstyle=\sl\ttfamily\scriptsize,breaklines=True]
# On branch master
#
# Initial commit
#
# Untracked files:
#   (use "git add <file>..." to include in what will be committed)
#
#	myfile.txt
nothing added to commit but untracked files present (use "git add" to track)
\end{lstlisting}
\end{block}
\end{frame}





\begin{frame}[fragile]
\begin{block}{Make a commit}
\begin{lstlisting}[language=bash,basicstyle=\ttfamily\scriptsize]
% git add myfile.txt
% git status
\end{lstlisting}

\begin{lstlisting}[language=bash,basicstyle=\sl\ttfamily\scriptsize,breaklines=True]
# On branch master
#
# Initial commit
#
# Changes to be committed:
#   (use "git rm --cached <file>..." to unstage)
#
#	new file:   myfile.txt
\end{lstlisting}


\begin{lstlisting}[language=bash,basicstyle=\ttfamily\scriptsize]
% git commit -m 'my first commit'
\end{lstlisting}

\begin{lstlisting}[language=bash,basicstyle=\sl\ttfamily\scriptsize,breaklines=True]
[master (root-commit) 4fe00ad] my first commit
 1 files changed, 3 insertions(+), 0 deletions(-)
 create mode 100644 myfile.txt
\end{lstlisting}

\end{block}
\end{frame}





\begin{frame}[fragile]
\begin{block}{Look at status}
\begin{lstlisting}[language=bash,basicstyle=\ttfamily\scriptsize,breaklines=true]
% git status
\end{lstlisting}
\begin{lstlisting}[language=bash,basicstyle=\sl\ttfamily\scriptsize]
# On branch master
nothing to commit (working directory clean)
\end{lstlisting}
\end{block}

\begin{block}{Make some changes, and a new file}
\begin{lstlisting}[language=bash,basicstyle=\ttfamily\scriptsize,breaklines=true]
% gedit myfile.txt
% git status (output not shown)
% gedit anewfile.txt
% git status
% git add anewfile.txt
% git commit -a -m 'my second commit'
\end{lstlisting}

\begin{lstlisting}[language=bash,basicstyle=\sl\ttfamily\scriptsize]
2 files changed, 2 insertions(+), 0 deletions(-)
create mode 100644 anewfile.txt
\end{lstlisting}

\end{block}



\begin{block}{Make a few more commits}
\begin{lstlisting}[language=bash,basicstyle=\ttfamily\scriptsize,breaklines=true]
% gedit / git add / git commit / ...
\end{lstlisting}
\end{block}
\end{frame}


\begin{frame}[fragile]
\begin{block}{Browsing history using GUI tools}
\begin{lstlisting}[language=bash,basicstyle=\ttfamily\scriptsize,breaklines=true]
% git log 
% gitg
\end{lstlisting}

\end{block}
\end{frame}





\begin{frame}[fragile]
\begin{block}{Jumping back to an old commit}
\begin{lstlisting}[language=bash,basicstyle=\ttfamily\scriptsize,breaklines=true]
% git log
% git checkout <first-chars-of-sha1-hash>
% git checkout 4fe00ad2b950
% ls
\end{lstlisting}

\begin{lstlisting}[language=bash,basicstyle=\sl\ttfamily\scriptsize]
myfile.txt
\end{lstlisting}
\end{block}
\end{frame}









%\begin{frame}[fragile]
%\begin{block}{Adding some files}
%\begin{lstlisting}[language=bash,basicstyle=\ttfamily\scriptsize,breaklines=true]
%% git status
%\end{lstlisting}
%\begin{lstlisting}[language=bash,basicstyle=\sl\ttfamily\scriptsize]
%\end{lstlisting}
%\end{block}
%\end{frame}



 
\begin{frame}
\begin{block}{Brief discussion of concept of branching}
\end{block}
\end{frame}


\begin{frame}
\begin{block}{Brief discussion of concept of syncing multiple computers}
Pushing and Pulling
\end{block}
\end{frame}


\begin{frame}
\begin{block}{Online git book}
\url{http://git-scm.com/book/en/Git-Basics}
\end{block}
\end{frame}

\begin{frame}
\begin{block}{Any Questions}
\end{block}
\end{frame}

\end{document} 
