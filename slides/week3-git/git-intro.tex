\documentclass{beamer}
%\usetheme{Madrid} % My favorite!
%\usetheme{Boadilla} % Pretty neat, soft color.
%\usetheme{default}
%\usetheme{Warsaw}
%\usetheme{Bergen} % This template has nagivation on the left
%\usetheme{Frankfurt} % Similar to the default 
%with an extra region at the top.
%\usecolortheme{seahorse} % Simple and clean template
%\usetheme{Darmstadt} % not so good
% Uncomment the following line if you want %
% page numbers and using Warsaw theme%
% \setbeamertemplate{footline}[page number]
%\setbeamercovered{transparent}
\usetheme{Goettingen}

\setbeamercovered{invisible}
% To remove the navigation symbols from 
% the bottom of slides%
\setbeamertemplate{navigation symbols}{} 
%
\usepackage{graphicx}
\usepackage{verbatim}
%\usepackage{bm}         % For typesetting bold math (not \mathbold)
%\logo{\includegraphics[height=0.6cm]{yourlogo.eps}}
%

\title[A brief introduction to Git]{A brief introduction to Git}
\author{Michael Hull}
\institute[University of Edinburgh]
{
University of Edinburgh \\
\medskip
{\emph{mikehulluk@googlemail.com}}
}
\date{\today}
% \today will show current date. 
% Alternatively, you can specify a date.
%
\begin{document}
%
\begin{frame}
\titlepage
\end{frame}
%



\begin{frame}
\frametitle{Overview}
\begin{block}
{This talk}
\begin{itemize}
\item Overview of Git (30 mins)
\item Practical Session (30 mins)
\end{itemize}
\end{block}
\end{frame}



\begin{frame}
\frametitle{How do you ...}

\begin{block}
{ ... keep your Laptop \& Desktop in-sync}
\begin{itemize}
\item Dropbox?
\item E-mail? (Google Drive)
\item USB stick
\item (Unison/rsync/ssh)
\item ??
\end{itemize}
\end{block}

\pause

\begin{block}
{... backup your code}
\begin{itemize}
\item \emph{(Methods above)} 
\item Time Machine 
\item Copy to university-server
\item ??
\end{itemize}
\end{block}

\end{frame}



\begin{frame}
\frametitle{How do you ...}
\begin{block}
{... keep old copies of your code}	
\begin{itemize}
\item Create 'zip' files of the directories with the date. 
\item ??
\end{itemize}
\end{block}

\pause

\begin{block}
{ \emph{Version Control systems solve these problems (\& more)}	}
\end{block}




\end{frame}


\begin{frame}
\frametitle{Version Control}

\begin{block}
{History}
\begin{itemize}
\item Allow multiple developers to work on the same code (1970's)
\item Designed by programmers for programmers - work well with text-files
\item Commandline \& graphical tools
\end{itemize}
\end{block}

\end{frame}


\begin{frame}
\frametitle{Version Control}


\begin{block}
{Common Concepts}
\begin{itemize}
\item Save \emph{snapshots} of a directory at a point in time, which is associated with a version number. Each snapshot is called a {\bf{commit}}. 
\item Allow you to see the changes made between particular snapshots.
\item Allow you to jump in time to different snapshots - {\bf{check-out}}. 
\item Allow you to {\bf{branch}} and {\bf{merge}}.
\end{itemize}
\end{block}
\end{frame}


\begin{frame}
\frametitle{Centralised vs Distributed}
\begin{block}
{2 types}
\begin{description}
\item[Centralised - \emph{SVN (subversion)} ] The project is stored on a central server, users communicate with the server to make snapshots, jump backwards to a previous commit, etc.
\item[Distributed - \emph{git (mercurial,...)} ] Each developer has an entire copy of the project on thier local machine. Changes are 'pushed' and 'pulled' between copies of the project. (Often a central repository is used)
\end{description}
\end{block}
\end{frame}
%
\begin{frame}
\frametitle{Git}
\begin{block}{}
\begin{itemize}
\item Modern Distributed system (2005)
\item Developed to manage the development of the Linux-Kernel. (Last week 1 commit every 2.5 minutes.)
\item Steep learning curve (there are GUI tools)
\item (git can 'talk-to' SVN repositories)
\item (Windows/Mac/Linux/etc)
\end{itemize}
\end{block}
\end{frame}


\begin{frame}[fragile]
\frametitle{Git Commands}
%
\begin{block}{Initial Setup}
\begin{itemize}
\item \verb|git init| - used once to start a new project
\item \verb|git clone <location>| - clone an existing project
\end{itemize}
\end{block}


\begin{block}{Status}
\begin{itemize}
\item \verb|git status| - show the status of the files in the repository
\end{itemize}
\end{block}

\end{frame}


\begin{frame}[fragile]
\frametitle{Git Commands}

\begin{block}{Making Commits}
\begin{itemize}
\item \verb|git add <filename>| - select files to add to the next commit (snapshot)
\item \verb|git commit -m '<message>'| - make a commit
\end{itemize}
\end{block}


\begin{block}{Syncronising Commits}
\begin{itemize}
\item \verb|git pull <location>| - update the local repository with commits made in a remote repository.
\item \verb|git push <location>| - update a remote repository with commits made in this local repository.
\end{itemize}
\end{block}
\end{frame}


\begin{frame}[fragile]
\frametitle{Git Tools}

\begin{block}{Graphical Tools}
\begin{itemize}
\item gitg
\item git cola
\end{itemize}
\end{block}
\end{frame}






\begin{frame}
\frametitle{Git Examples}
\begin{block}{Demo...}
\end{block}
\end{frame}





 
\begin{frame}
\centerline{The End}
\end{frame}




\end{document} 
